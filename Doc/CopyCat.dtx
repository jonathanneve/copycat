
@@Replication nodes
<autolink ON>

A replication node is simply a database that is involved in the replication. If,
for instance, you set up replication between one central server and two branch
offices, there will be three nodes, one for each database involved.<p/>
<p/>
Each of these nodes must be given a name that will be used within CopyCat to
identify which database the changes come from, so that the changes that are
replicated to one node are not bounced back to the originating node.<p/>
<p/>
So, in each database involved in the replication, you have to fill the <link RPL$USERS>
table with the list of nodes towards which this database is to send its changes.
Every change that is made to a replicated table, triggers one new line in <link RPL$LOG>
for each of these nodes.<p/>
<p/>
Then, in TCcReplicator, you have to set the names of the two nodes you are
replicating between. Typically, the local node would be the branch office, and
the remote node would be the central server (and it must be this way if you want
to use CopyCat's primary key synchronization), but it could be the other way
around.<p/>
<p/>
TCcReplicator uses the node names that you give it (via the LocalNode.Name and
RemoteNode.Name properties) when connecting to the databases. The remote node is
queried to see if there are any new changes to replicate for the local node, and
vice-versa.<p/>
<p/>
This means that if you want bi-directional replication, you need to make sure the
name of the local node is present in the RPL$USERS table of the remote node, and
vice versa. If on the other hand you only want the changes from the remote
database to be sent to the local database, you should put the local node name in
the remote node's RPL$USERS, but exclude the remote node from the local node's
RPL$USERS.<p/>
<p/>
For example, suppose a replication setup had one main server (called user 'S'),
and two branch offices (called users 'A' and 'B'). For full two-way replication,
S would need to have both A and B in its <link RPL$USERS> table, while A and B
would each have S in their's. When a data change occurs in S, it triggers two
lines in <link RPL$USERS, RPL>$<link RPL$USERS, USERS>, one for A, and one for B.<p/>
<p/>
The next time you replicate between A and S, TCcReplicator will fetch from <link RPL$LOG>
all the records that concern node A, without interfering with node B in any way.
B may replicate before, after, or simultaneously, without it making any
difference whatsoever.<p/>
<p/>
Suppose on the other hand, that a data change takes place in database A. This
will trigger a line in A's <link RPL$LOG> table, addressed to node S. When you
replicate between A and S, TCcReplicator fetches from the local database (A) all
the records in <link RPL$LOG> that concern the node S. It then replicates the
change to database S.<p/>
<p/>
This data change in S triggers a new line in S's <link RPL$LOG> table, addressed
to node B. When B then connects to S, it gets A's data change. Since (from B's
viewpoint) the data change came from S, it does not replicate back to S.
See Also:
<link RPL$USERS>, TCcReplicator.Nodes

@@Introduction
* Welcome to Microtec CopyCat ! *

CopyCat is designed to provide Delphi, C++Builder and FreePascal developers an
all-in-one solution for database replication needs. It is comprised of the
following components:<p/>


<table noborder, c%, 70c%>
<image TCCREPLICATORimg>   Core replication component � TCcReplicator handles full two-way
 <b>TCcReplicator</b>       replication between two databases.
<image TCcConfigimg>       Configuration component � Use TCcConfig to create in each database
 <b>TCcConfig</b>           the meta-data necessary for replication.
<image TCCQUERYimg>        Use TCcQuery to execute queries using CopyCat's connection system.
 <b>TCcQuery</b>            A TCcQuery can be hooked up to any type of TCcConnection, and can
                            therefore execute queries through any direct connector (FIBPlus,
                            IBX, ADO, etc...) or through any transport (RTC, XMLRPC, ...).
<image TCCDATASETimg>      TCcDataSet provides an abstract TDataSet descendent, which can be
 <b>TCcDataSet</b>          linked up to data-aware controls for display and/or editing data.<p/>Like
                            TCcQuery, TCcDataSet can be used with any TCcConnection.<p/>The
                            TCcDataSet is not available on the mobile platforms nor on
                            FreePascal.
<image TCCMEMORYDATAImg>   TCcMemoryData is the memory dataset that is used internally in
 <b>TCcMemoryData</b>       CopyCat, as a basis for TCcDataSet. It is registered on the palette
                            merely for convenience. Internally, it is based on a TRxMemoryData.
                            It is not available on the mobile platforms nor on FreePascal.
</table>

CopyCat also registers a palette called "CopyCat connectors", which contains
\links to third-party connection components which CopyCat can use to establish a
database connection for replication.<p/>
<p/>
The first thing you need to do after installing CopyCat is to choose which
connectors you would like to use, and compile and install them into your IDE.
Each connector supported by CopyCat provides a bridge with a third-party database
connectivity library (IBX, IBObjects, FireDAC, UniDAC, ADO, etc.). There are also
transport connectors (RTC and XMLRPC), that enable CopyCat to replicate over a
middleware, which allows for improved security and performance.<p/>

Version:
<b>%PROJECTVERSION%</b>
Author:
%AUTHOR%. Feel free to <extlink http://copycat.fr/wordpress/contact-us>contact us</extlink>
if you have any feedback or questions.<p/>
<i>%COPYRIGHT%</i><p/>

@@Key design aspects
Below are a few core aspects of CopyCat, with a brief explanation of how it works
internally.

* 1.Fault-tolerance *

CopyCat creates for each table that need to be replicated a set of triggers,
which insert a line into the <link RPL$LOG> table every time a change occurs.
Contrary to some replicators, CopyCat does not indicate in the log the type of
operation (INSERT, UPDATE or DELETE), nor the actual data of the record. Rather,
it simply provides the primary key value(s) (up to a maximum of 3), so as to
indicate that the record has been "touched" in some way. This provides for much
flexibility.<p/>
<p/>
When replication starts, CopyCat reads in the list of changes from <link RPL$LOG>.
For each change in the log, it checks in the local and remote databases to see if
the record still exists. Based on this information, it can then issue an INSERT,
UPDATE or DELETE statement, as appropriate. Since CopyCat takes the current state
of the record into consideration, rather than the actual changes that took place,
it is highly fault-tolerant.<p/>
<p/>
For example, suppose a record was inserted into the local database, and, for
whatever reason, it wasn't initially replicated to the remote database (this
could only happen if the meta-data wasn't configured properly, or if the node
wasn't in <link RPL$USERS>). Once the configuration problem has been resolved, if
another change is made to the record (an UPDATE statement), it will get correctly
INSERTed into the remote database. Thus, in reality, on the local database, an
UPDATE statement was issued, but when the change gets replicated, an INSERT
statement is issued to the remote database, since the record doesn't exist yet.

* 2.Scalability *

Each database holds a list of nodes it's replicating for, and can thus be
configured to replicate its changes to one or several remote databases. The
simplest form of database replication would involve both local and remote
databases simply replicating to each other. But it is easy to create more complex
replication structures than that.<p/>
<p/>
A simple example would be a central server, replicating for several remote
clients. Every change made on the server would be replicated to each remote
client, and the changes made on the remote clients would be replicated towards
the server, and, indirectly, towards all the other remote clients as well.<p/>
<p/>
For larger scale deployments, it would be easy to have several servers, each with
its set of clients replicating to and from it, and each replicating all the
changes to another central server, or even one of several redundant replication
servers (if the load is too high to allow having one central server).

* 3.Ease of setup *

CopyCat has been designed to require very little initial setup, and, most
importantly, requires no architectural changes to your databases nor to your
applications. Even very complex databases can be replicated using CopyCat,
usually in a completely transparent fashion. This is made possible in particular
by the use of two of CopyCat's most unique features: primary key synchronization
(using either a generator, a stored procedure, or any other custom SQL
expression), and stored procedure "replication" (allowing to simply echo all
calls to a stored procedure directly to the remote database).

* 4.Meta-data changes *

The CopyCat triggers only need to be regenerated if the primary key(s) of a table
change, or if a new table needs to be replicated. Adding or removing fields can
be done without needing any reconfiguration.<p/>
<p/>
CopyCat can also be configured to be harmonize the list of field for each table
before replicating. That means that if a field is present in one database but not
in the other, it will be omitted from the replication process, rather than
produce an error.
See Also:
TCcReplicator, TCcConfig

@@Conflict management
CopyCat features a set of advanced conflict management techniques.<p/>
<p/>
Before replication starts, both the local and the remote replication logs (that
is, database table <link RPL$LOG>) are checked for conflicts. If a conflict is
found, that is, if a certain record has been modified in both databases since the
last replication, the conflict is reported, and the record is not replicated in
either direction until it is resolved. The record will remain unsynchronized
until the conflict is resolved: <b>thus, there can never be any data loss.</b><p/>
<p/>
There are two levels of conflict management:

  1. First, the TCcReplicator.OnResolveConflict event is fired, allowing the
     developer to implement customized conflict resolution policies as applicable. A
     TConflictRecord containing all the information about the conflict is passed as a
     parameter to this event, so that you have all the information you need for
     resolving the conflict at your fingertips. To resolve the conflicte, set
     TConflictRecord.ChosenNode to the name of the node whose changes you would like
     to keep, and the record will directly be replicated accordingly.
  2. If The OnResolveConflict is not implemented or if no node is chosen, the
     conflict is logged to <link RPL$CONFLICTS>. The <link RPL$CONFLICTS> table copies
     the main fields of <link RPL$LOG> (so as to be able to identify the conflictual
     record), and adds a few of its own, namely CONFLICT_DATE (date when the conflict
     was first logged), USER1 and USER2 (the two <link Replication nodes, replication nodes>
     the conflict is between), and CHOSEN_USER (the user whose version of the record
     should be kept). CHOSEN_USER is blank by default, it's up to the application user
     to resolve the conflict by choosing either USER1 or USER2 as having the correct
     \version of the record, and putting the chosen user's name into the CHOSEN_USER
     field. Upon the next replication, TCcReplicator will check the conflict log: if a
     conflict has been resolved (that is, if CHOSEN_USER is not blank), the correct
     \version of the record will be replicated to the other side, and the log of the
     conflict will be removed.
See Also:
<link RPL$CONFLICTS>, TCcConflictMgr

@@RPL$USERS
The RPL$USERS table holds the list of all the nodes the database is to replicate
towards. Until this table is filled, no data changes will be logged in <link RPL$LOG>.
It has the following fields:

<table 25c%, 15c%>
Field name        Data type   \Description
================  ==========  -----------------------------------------------------------------
LOGIN             String      Name of the replication node
LIBELLE           String      Label describing the node. Not used internally by CopyCat, it is
                               merely supplied for convinence.
CONDITION_VALUE   String      Spare field for putting user-defined data or ID. Used for
                               matching with RPL$TABLES.CONDITION_FIELD.
CONFIG_NAME       String      Name of the configuration this node is attached to.�Corresponds
                               to TCcConfig.ConfigName.
PASSWRD           String      Password, used only for logging into CopyCat transport servers
</table>
See Also:
<link Replication users>

@@RPL$CONFLICTS
The RPL$CONFLICTS table is a log of all the currently conflicting data changes in
the database. It has the following fields:

<table 25c%, 15c%>
Field name               Data type    \Description
=======================  ===========  ----------------------------------------------------
CODE<p/>                 Integer      Integer code uniquely identifying each conflict.
USER1<p/>                String<p/>   Replication nodes involved in the conflict.
USER2<p/>                String<p/>   
CHOSEN_USER<p/>          String<p/>   Replication node chosen as having the correct
                                       \version of the record. By default, this field is
                                       blank.
CONFLICT_DATE<p/>        Date/Time    Date and time when the conflict was first recorded.
TABLE_NAME<p/>           String<p/>   Name of the table.
REMOTE_TABLE_NAME        String       Name of the table in the remote database
PRIMARY_KEY_VALUES<p/>   String<p/>   Primary key values.
REF_VALUE<p/>            String       Deprecated -- don't use for now.
SQL1                     Memo         Names and values of the fields on node 1
SQL2                     Memo         Names and values of the fields on node 2
CHANGED_FIELDS1          Memo         List of fields changed by node 1
CHANGED_FIELDS2          Memo         List of fields changed by node 2
</table>
See Also:
<link Conflict management>

@@Database tables
CopyCat requires and creates the following database tables:

  1. <link RPL$LOG> � A log of every data change that takes place in the
     database.
  1. <link RPL$LOG_VALUES> � The values of each field changed � only used if
     TCcConfig.TrackChangedFields option is set.
  2. <link RPL$USERS> � The list of all the nodes the database is to replicate
     towards.
  3. <link RPL$CONFLICTS> � All the currently conflicting data changes.
  4. <link RPL$TABLES> � The list of all tables in the database, with replication
     options for each one.
  5. <link RPL$PROCEDURES> � The list of all non-selectable procedures in the
     database, with replication options for each one.

@@RPL$LOG
The RPL$LOG table is a log of every data change that takes place in the database.
One line is created per user in <link RPL$USERS>. It has the following fields:

<table 25c%, c%, 65c%>
Field name            Type             \Description
====================  ===============  -----------------------------------------------------------------------
<align center>        <align center>   Table name
 TABLE_NAME            String           
 </align>              </align>         
<align center>        <align center>   Values of the primary key fields
 PRIMARY_KEY_VALUES    String           
 </align>              </align>         
<align center>        <align center>   Synchronization statements for each of the
 PRIMARY_KEY_SYNC      String           primary key fields These are SQL statements
 </align>              </align>         that can be used for fetching unique primary
                                        key values from the remote server.
<align center>        <align center>   Synchronization statements for each of the
 UNIQUE_KEY_SYNC       String           unique key fields. These are SQL statements
 </align>              </align>         that can be used for fetching unique field
                                        values from the remote server.
<align center>        <align center>   SQL of the EXECUTE PROCEDURE statement that is
 PROCEDURE_STATEMENT   String           to be executed in order to replicate a stored
 </align>              </align>         procedure.
<align center>        <align center>   <link Replication nodes, Replication node>
 LOGIN                 String           that this line is addressed to.
 </align>              </align>         
<align center>        <align center>   Date and time of the data change. This field
 OPERATION_DATE        Date/Time        is merely informational, and is not used
 </align>              </align>         internally by CopyCat.
<align center>        <align center>   'Y' or 'N', to indicate whether a <link Conflict management, conflict>
 CONFLICT              String           has been logged for this line or not.
 </align>              </align>         
<align center>        <align center>   Integer code uniquely identifying the log
 CODE                  Integer          line.
 </align>              </align>         
SENT_FROM             String           <link Replication nodes, Replication node>
                                        that this line originates from.
OPERATION_TYPE        String           I, U or D to indicate whether it was an
                                        insert, an update or a delete
CHANGE_NUMBER         Integer          Unique number identifying the change
</table>
See Also:
TCcLog

@@RPL$TABLES
The RPL$TABLES table holds the list of all the non-system tables in the database,
with the replication configuration options for each one.<p/>
<p/>
RPL$TABLES has the following fields:

<table 17c%, 15c%>
Field name         Data type   \Description
-----------------  ----------  -----------------------------------------------------------------------
TABLE_NAME         String      Name of the table.
TRIG_BASE_NAME     String      Base name for the triggers. To this will be added '_I', '_U' or '_D'
                                depending on the type of trigger. By default, the base name is 'RPL$',
                                followed by as much of the table name as will fit. This might need to
                                be changed for long table names, in order to have unique trigger
                                names.
PRIMARY_KEY_SYNC   String      Synchronization statements for the each primary key field. These are
                                SQL statements that can be used for fetching unique primary key values
                                from the remote server.
UNIQUE_KEY_NAMES   String      Names of any other fields for which global unicity is to be
                                maintained.
UNIQUE_KEY_SYNC    String      Synchronization statements for each of the unique key fields. These
                                are SQL statements that will be used for fetching unique field values
                                from the remote server.
REPL_UPDATES       String      Determines whether updates should be replicated or not.
                    (Y/N)       
REPL_INSERTS       String      Determines whether inserts should be replicated or not.
                    (Y/N)       
REPL_DELETES       String      Determines whether deletes should be replicated or not.
                    (Y/N)       
CONDITION          String      SQL condition determining in which cases and to which nodes a row is
                                to be replicated. This condition will be placed as is in the trigger
                                SQL. To access the table fields from within this condition, use the
                                "new." prefix, and to access the RPL$USERS table (in order to
                                determine which nodes will receive changes from this table), use the
                                "u." prefix.<p/><p/>For example, setting the condition "new.a_field \>
                                4" means that no changes will be replicated until the value of a_field
                                is greater than 4. Setting the condition to "new.dest_user = u.login"
                                means that only the node with login equal to the dest_user field will
                                receive the changes from this table.
CONDITION_FIELD    String      Name of a field in the table to be matched against
                                RPL$USERS.CONDITION_VALUE. Setting CONDITION_FIELD to "field_name" is
                                equivalent to setting the CONDITION field to "u.condition_value =
                                new.field_name".
PRIORITY           Integer     Priority of the table. This value is designed to avoid problems
                                arising from inter-table dependencies. Each table should have a higher
                                priority than any of the other tables it depends on, so as to ensure
                                that it is replicated after them.
CREATED            String      Indicates whether of not the table's replication triggers have been
                    (Y/N)       generated (either 'Y' or 'N').
REF_FIELD          String      Deprecated, don't use for now.
REF_TABLE          String      Deprecated, don't use for now.
REF_TABLE_KEY      String      Deprecated, don't use for now.
</table>

@@RPL$PROCEDURES
The RPL$PROCEDURES table holds the list of all the executable (that is,
non-selectable) procedures in the database, with the replication configuration
options for each one.<p/>
<p/>
RPL$PROCEDURES has the following fields:
<table 20c%, 10c%>
Field Name           Data type   \Description
-------------------  ----------  ----------------------------------------------------------------------
PROCEDURE_NAME       String      Name of the procedure.
NEW_PROCEDURE_NAME   String      Base name for the replication procedure. By default, the base name is
                                  'RPL$', followed by as much of the procedure name as will fit. This
                                  might need to be changed for long procedure names, in order to keep
                                  unicity.
PRIORITY             Integer     Priority of the procedure. This value is designed to avoid problems
                                  arising from dependencies upon tables or other procedures. Each
                                  procedure should have a higher priority than any of the tables or
                                  procedures it depends on, so as to ensure that it is replicated after
                                  them.
CREATED              String      String Indicates whether of not the procedure's replication procedure
                                  has been generated (either 'Y' or 'N').
</table>


@@RPL$ERRORS
The RPL$ERRORS table holds the list of rows that failed to replicate last
replication session, and stores the error message and context information. This
table is emptied at every replication, meaning that only errors that are not
resolved during the next replication will appear in this table.<p/>
<p/>
RPL$ERRORS has the following fields:
<table 20c%, 10c%>
Field Name           Data type   \Description
-------------------  ----------  ------------------------------------------------------------
TABLE_NAME           String      Name of the table.
CODE                 Integer     Unique ID
PRIMARY_KEY_VALUES   String      Values of the primary key field(s).
ERROR_MESSAGE        Memo        Full error message, including offending SQL statement and
                                  its parameter values.
ERROR_CONTEXT        String      String describing which operation was taking place when the
                                  failure occured (SELECT, UPDATE, INSERT, DELETE,
                                  CHECKRECORDEXISTS)
</table>

@@RPL$TRACE
The RPL$TRACE table holds a trace of all the SQL statements that are applied
databases. This log is activated by setting the TCcReplicator.TraceType property.
Warning : this can take a lot of space, so tracing should either be used only
occasionally, for debug purposes, or else care should be taken to empty RPL$TRACE
regularly.<p/>
<p/>
RPL$TRACE has the following fields:
<table 20c%, 10c%>
Field Name           Data type   \Description
-------------------  ----------  ------------------------------------------------------------
TABLE_NAME           String      Name of the table.
CODE                 Integer     Unique ID
PRIMARY_KEY_VALUES   String      Values of the primary key field(s).
FROM_NODE            String      Name of the node from which this row originated.
TO_NODE              String      Name of the node to which this row is to be replicated.
OPERATION_DATE       Date        Date at which the original operation occured in the
                                  originating database. This corresponds to the date recorded
                                  in RPL$LOG.OPERATION_DATE.
REPLICATION_DATE     Date        Date and time at which the row was replicated and the SQL
                                  executed.
SQL_STATEMENT        Memo        Full SQL executed, containing full values for all
                                  \parameters.
</table>

@@Getting started
<b>Getting started with CopyCat</b><p/>
<p/>
CopyCat supports two different replication modes that can be used, depending on
the situation.<p/>
<p/>
<b>Standard</b><p/>
<p/>
This is the default mode, and it versions prior to 3.08.0, it was the only mode.<p/>
<p/>
In this mode, CopyCat creates triggers that simply log a row in RPL$LOG every
time a change occurs in one of the tables being replicated. The type of operation
(INSERT, UPDATE�or DELETE) is not recorded, because it is determined at
replication time. The values of the fields are also not recorded, only the
primary key, and therefore, there is no way of knowing which field has changed,
if any.<p/>
<p/>
At replication time, the row is then selected from the source DB. If it can't be
found, it is deleted from the destination DB. If it is found, then an update is
attempted. If the update finds no row, then it is inserted.<p/>
<p/>
This mechanism is robust and fault-tolerant, because even if the databases get
out of sync for whatever reason, it will still work (for example, if a row in the
source DB gets updated, that update can be turned into an insert if it turns out
that the row doesn't exist in the destination DB at replication time).<p/>
<p/>
It also has the advantage that if the same row modified several times, the rows
can be bunched up and replicated only once.<p/>
<p/>
This mode doesn't make it possible to guarantee consistency however, because the
exact order of events cannot always be reproduced: if you insert a row into table
A, setting only a few fields, then insert a row into table B, and then update
table A again, giving a value to a field with a foreign key pointing to table B,
then the insert into table A will fail at replication time, because it will try
to execute the full insert of all the fields, including the field with the
primary key, before the insert into table B, so there will be a foreign key
violation.<p/>
<p/>
The result of that is that there can be an inconsistent view of the data, it will
not be replicated atomically. For some uses, that doesn't really matter, but for
reporting or other such situations where it's more important for the data to be
consistent than fully up-to-date, that can be a problem.<p/>
<p/>
The other problem with this mode is that since all fields of the table are
replicated every time there is a change, there can be a performance hit for
tables that have large numbers of fields and/or big blob fields, as the values of
these fields will be sent over the wire every time the row changes, even if those
fields didn't change.<p/>
<p/>
<b>High-consistency with field change tracking</b><p/>
<p/>
This is a new feature of CopyCat 3.08.0, and is currently available only for
Firebird and Interbase, though it will be implemented for all other supported
databases in the upcoming releases.<p/>
<p/>
This mode enables a full tracking of the values of the fields that were changed
with every change to the database. That way, when it comes time to replicate, we
can know exactly which fields were changed and what values were given, and thus,
we can play back precisely the changes that took place in the source database, in
exactly the same order, and therefore, the foreign key issues mentioned above no
longer apply.<p/>
<p/>
The drawback is that it is entirely incremental, and will not work at all if the
two databases become out of sync for whatever reason. To fix such a situation
would require a pumping of the entire table or database to ensure that they are
identical.<p/>
<p/>
The major advantage however is better consistency and therefore improved
reliability. It is HIGHLY RECOMMENDED that you run the entire replication cycle
using a single transaction, and that you rollback all the changes if ever there
is an error on any row: this is important because some of the later changes in
the list may fail and it could lead to a situation where there are subtle
differences between the two databases. The only reliable way to use this feature
is to replicate atomically.<p/>
<p/>
Consistency is a huge advantage in many situations, because it means that either
the replication failed, in which case you will see none of the changes, or it
succeeded, in which case they will all be visible, but you can count on there
never being anything missing, and you can also be sure that there is no
incoherent data in the database (an invoice header with no lines for example).<p/>
<p/>
<b>Conflict resolution</b><p/>
<p/>
The other big advantage to this mode is finer grained and easier to manage
conflict resolution.<p/>
<p/>
Since we know which fields changed, many conflicts can be automatically resolved
by merging the changes from both nodes, in cases where none of the same fields
were changed on both sides. In cases that are still conflictual, you are provided
the list of fields changed on each side as well as the list of fields changed on
both sides, and it is also possible to find out for each field the value that it
was changed from and to on each side, making conflict resolution much easier.<p/>
<p/>
If a conflict is detected, the fields that were changed on only one of the nodes
can be merged. For instance, if the first node changed fields A, B and C while
the seconde node changed fields C and D, if the conflict is resolved in favor of
the second node, the values of A and B from the first node will be kept, as they
were not conflictual with the change from the second node. Thus, conflict
resolution can be much finer-grained.<p/>
<p/>
For more information about conflict management, see: 

@@RPL$LOG_VALUES
The RPL$LOG_VALUES table contains the list of all the fields changed, with their
values. It has the following fields:

<table 25c%, c%, 65c%>
Field name       Type             \Description
---------------  ---------------  -------------------------------------------
<align center>   <align center>   <link Replication nodes, Replication node>
 NODE_NAME        String           this change is logged for
 </align>         </align>         
<align center>   <align center>   Number of the change: corresponds to
 CHANGE_NUMBER    Integer          the CHANGE_NUMBER field of RPL$LOG
 </align>         </align>         
<align center>   <align center>   Old value of the field, stored as a
 OLD_VALUE        String           varchar(200), if it fits
 </align>         </align>         
<align center>   <align center>   Old value of the field, stored as a
 OLD_VALUE_BLOB   Blob             blob, for blob values and for values
 </align>         </align>         larger than 200 characters
<align center>   <align center>   New value of the field, stored as a
 NEW_VALUE        String           varchar(200), if it fits
 </align>         </align>         
<align center>   <align center>   New value of the field, stored as a
 NEW_VALUE_BLOB   Blob             blob, for blob values and for values
 </align>         </align>         larger than 200 characters
<align center>   <align center>   Name of the field
 FIELD_NAME       String           
 </align>         </align>         
<align center>   <align center>   Data type of the field
 FIELD_TYPE       String           
 </align>         </align>         
</table>
See Also:
TCcLog
