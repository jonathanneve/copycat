
@@CcReplicator.pas
CopyCat replication suite                     
                                              
Copyright (c) 2005 Microtec Communications    
                                              
For any questions or technical support,       
contact Jonathan Neve (jonathan@microtec.fr)

@@TAutoRepl
Summary:
Defines automatic replication properties.
Description:
Use TAutoRepl to automatically launch a replication session periodically, if
there is none already in progress.<p/>
<p/>
To enable automatic replication, set Enabled to True, and call the Start method
when you are ready to start the timer. Automatic replication can be temporarily
disabled at any time, by calling the Stop method.
See Also:
TCcReplicator.AutoReplicate

@@TAutoRepl.Enabled
Summary:
Enables or disables automatic replication.
Description:
Set Enabled to enable or disable automatic replication is a general way. If
auto-replication is enabled, the Start method must still be called however, in
order to start the timer.

@@TCcAutoCommit
Summary:
Represents auto-commit properties.
Description:
The TCcAutoCommit object allows to periodically commit the transactions after
Frequency seconds of replication. Use the CommitType property to select the type
of commit that you want to perform.<p/>
<p/>
It is not necessary to call the Start nor Stop methods, as this is handled
internally be TCcReplicator. Setting CommitType to ctNone means that no automatic
commit will be performed.
See Also:
TCcReplicator.AutoCommit

@@TCcAutoCommit.CommitType
Summary:
Type of automatic commit to perform.
Description:
Set CommitType to determine how you wish to commit the replication transactions.

@@TCcCustomLog.Bof
Bof is true on the first record of the log

@@TCcCustomLog.CurrentLine
Current line number

@@TCcCustomLog.dpDest
Overview: Destination database for the current record.
Description: dpDest corresponds to the destination database of the current record, that is,
  the database towards which the current record is to be replicated.
SeeAlso TCcCustomLog.dpOrigin

@@TCcCustomLog.dpOrigin
Summary:
Source database for the current record.
Description:
dpOrigin corresponds to the source database of the current
record, that is, the database from which the current record
originates. SeeAlso TCcCustomLog.dpDest

@@TCcCustomLog.Edit
Summary:
Edit the current record of the replication log.
Description:
Identical to its TDataSet counterpart.

@@TCcCustomLog.Eof
Eof is true on the last record of the log

@@TCcCustomLog.FBN@String
Summary:
Retrieve field value as a string.
Description:
FBN(FieldName) is equivalent to FieldByName(FieldName).AsString.

@@TCcCustomLog.FieldByName@String
Summary:
Retieve field object by name.
Description:
Use field name to access the TField object, in order to read or write the value
of a field.<p/>
Please note however, that application shouldn't normally change the values of any
fields of the replication log, and doing so may cause unpredicted misbehavior.
See Also:
FBN

@@TCcCustomLog.First
Summary:
Go to first record of the log.
Description:
Identical to its TDataSet counterpart.

@@TCcCustomLog.Last
Summary:
Go to last record of the log.
Description:
Identical to its TDataSet counterpart.

@@TCcCustomLog.LineCount
Total number of lines in the log

@@TCcCustomLog.LocalMode
Indicates whether the record comes from the local database.

@@TCcCustomLog.LogConflict
TCcReplicator

@@TCcCustomLog.Next
Summary:
Go to the next record of the log, if any.
Description:
If the current record is already the last, the Eof property is set to True.

@@TCcCustomLog.Post
Summary:
Post the current record of the replication log.
Description:
Identical to its TDataSet counterpart.

@@TCcCustomLog.Prior
Summary:
Go to previous record of the replication log.
Description:
Identical to its TDataSet counterpart.

@@TCcCustomLog.ReplState
Overview: State of the current record.
Description: The state may be either lsNone (non-replicated), lsOk (replicated),
  lsConflict (cannot be replicated due to a conflict), or lsError (an error occured while
  trying to replicate).

@@TCcLog.Edit
Summary:
Edit the current record of the replication log.
Description:
Identical to its TDataSet counterpart.

@@TCcLog.FBN@String
Summary:
Retrieve field value as a string.
Description:
FBN(FieldName) is equivalent to FieldByName(FieldName).AsString.

@@TCcLog.FieldByName@String
Summary:
Retieve field object by name.
Description:
Use field name to access the TField object, in order to read or write the value
of a field.<p/>
Please note however, that application shouldn't normally change the values of any
fields of the replication log, and doing so may cause unpredicted misbehavior.
See Also:
FBN

@@TCcLog.First
Summary:
Go to first record of the log.
Description:
Identical to its TDataSet counterpart.

@@TCcLog.Last
Summary:
Go to last record of the log.
Description:
Identical to its TDataSet counterpart.

@@TCcLog.Next
Summary:
Go to the next record of the log, if any.
Description:
If the current record is already the last, the Eof property is set to True.

@@TCcLog.Post
Summary:
Post the current record of the replication log.
Description:
Identical to its TDataSet counterpart.

@@TCcLog.Prior
Summary:
Go to previous record of the replication log.
Description:
Identical to its TDataSet counterpart.

@@TCcQueryType
Description:
Type of SQL query.
See Also:
TCcReplicator.OnQueryDone

@@TCcQueryType.qtDelete
Description:
DELETE query.

@@TCcQueryType.qtInsert
Description:
INSERT query.

@@TCcQueryType.qtSelect
Description:
SELECT query.

@@TCcQueryType.qtUpdate
Description:
UPDATE query.

@@TCcReplicator
Summary:
TCcReplicator provides the core replication functionality of
CopyCat.
Description:
\ \ 

<b>How to use TCcReplicator:</b>
  1. Set the DBProvider property to the data access component
     set of your choice.
  2. Fill in the LocalDB and RemoteDB properties with the
     connection parameters for each database.
  3. Provide the LocalSYSDBA, LocalDB, and RemoteDB usernames
     and passwords. LocalSYSDBA corresponds to the SYSDBA of the
     local server, while LocalDB and RemoteDB users correspond to
     the names attibuted to the replication sites. See Users for
     more information.
  4. Call the Replicate method.
  5. Optionnaly, implement a few of the events, in order to
     provide user feedback at every stage of the replication
     process.
  6. Optionnaly, you may set up automatic timer-based
     replication, using the AutoReplicate property.
  
<align justify>
Alternatively, the configuration of the replicator can be
held in an INI-file, using the ConfigStorage property to link
to the TCcConfigStorage component managing the configuration.
  
If an exception occurs during replication, an
OnReplicationError event is fired, where you can choose to
either abort the whole replication process, or to continue
trying to replicate the next record. The default behaviour
depends on the AbortOnError property.
  
If an exception occurs at any other point (e.g. during
database connection), an OnException event is fired, and
replication is aborted.
  
TCcReplicator frequently fires the OnProgress event, in order
to allow the application to refresh user interface, process
any pending user messages, and possibly, update a progress
bar. It is also possible to abort the replication at any
time, using the AbortReplication method. Refer to the
provided example projects for an illustration of the use of
most of the events and properties.
</align>
See Also:
AbortOnError, OnProgress

@@TCcReplicator.AbortOnError
Summary:
Determines whether or not to abort replication when an error occurs.
Description:
When an error occurs during replication, the OnReplicationError event is fired,
to determine whether replication may resume (i.e. skip the current record, and
and continue with the subsequent ones), or must be aborted. AbortOnError is used
as a default behaviour in case this event is not implemented.
See Also:
OnReplicationError

@@TCcReplicator.AbortReplication
Summary:
Call AbortReplication to stop the current replication process.
Description:
Once replication has started, you may abort it at any time using the
AbortReplication method. AbortReplication doesn't actually stop replicating
immediately (because the database component will in most cases be busy, with no
way of interrupting them). As soon as the replication process has been aborted,
the OnAbort event is called.
See Also:
OnAbort

@@TCcReplicator.AutoCommit
Summary:
Handles automatic, timer-based committing during replication.
Description:
AutoCommit allows you to enable/disable automatic committing. If CommitType \<\>
ctNone, the transactions will automatically be committed every Frequency seconds,
using the commit type specified in CommitType.<p/>
If the commit timer fires during replication of a record, the commit will only
actually be performed once the record has been fully replicated. This timer is
reset every time the Replicate method is called.
See Also:
Replicate

@@TCcReplicator.AutoReplicate
Summary:
Handles automatic, timer-based replication.
Description:
AutoReplicate allows you to enable/disable automatic replication. If Enabled is
true, the replication process will automatically be launched every Frequency
seconds, unless another process is already in progress.<p/>
<p/>
In order to start the auto-replication timer, you must call the Start method.
Thus, Enabled should be set to true to enable auto-replication is a general way,
and Start must be called when the application is ready to start auto-replicating.
Use Stop to end the auto-replication timer.
See Also:
TAutoRepl.Start, TAutoRepl.Stop, Replicate, Busy

@@TCcReplicator.BeforeReplicate
Summary:
Fired just before replication begins.
Description:
BeforeReplicate occurs as soon as the Replicate method is called or the
auto-replicate timer is fired, before any replication takes place.
See Also:
Replicate, AutoReplicate

@@TCcReplicator.Busy
Overview: Indicates whether the replicator component is available or not.
Description: Busy is true from the moment the Replicate procedure is called,
until replication is either finished or aborted.

@@TCcReplicator.Closing
Overview: Indicates that the replication in the process of closing.
Description: Closing is set to true whenever the user chooses to abort the replication
  process (whether by the AbortReplication method, or in response to an OnReplicationError
  event). As soon as an appropriate moment is found, replication then stops, and Closing
  is set back to false.

@@TCcReplicator.CommitOnFinished
Overview: Determines how the transaction is ended after replication.
Description: Possible options are Commit, CommitRetaining, or none.

@@TCcReplicator.Connect
Overview: Connect to both local and remote databases.
Description: Connect establishes a connection with both the remote and the local databases,
if there was no existing connection, and starts both transactions. If the database connections
and/or transactions were already active, no error is produced.

Before establishing the connection, the OnConnectLocal or the OnConnectRemote events are fired.

@@TCcReplicator.Disconnect
Overview: Disconnect local and remote databases.
Description: Call Disconnect to cut the connection to both local and remote databases.

If the transactions are active, Disconnect first commits them. After this,
the OnDisconnect event is fired, and the database connections are cut.

@@TCcReplicator.Finished
Overview: Indicates that replication is finished.

@@TCcReplicator.HarmonizeFields
Summary:
Harmonize field list between local and remote databases.
Description:
If HarmonizeFields is true, only the fields that exist in both local and remote
databases will be replicated. In order to find the list of fields, the local
database is queried and (if necessary) the remote database also. The OnGetFields
event is then fired, in order to let the developer change the list of fields.<p/>
This implies that in certain cases, CopyCat will issue an additionnal query
against the remote database to find the list of fields for each table. Therefore,
setting HarmonizeFields to false will slightly improve performance, but will
produce an exception if ever a field is missing.<p/>
Actually, if there are lines in the log coming from the remote database, there is
no need to explicitly fetch the list of fields from the remote database, since
the data will be SELECTed from it. This optimizes performance by avoiding (part
of the time) an extra query against the remote database, and it also means that
in such cases, the field list is always harmonized, regardless of the value of
the HarmonizeFields property.
See Also:
OnGetFields

@@TCcReplicator.KeepConnection
Overview: Keep the database connections open after replication ends.
Description: If KeepConnection is false, the connections will be closed after every call to
  Replicate, and reopened upon the next replication. If KeepConnection is true,
  the database connections will be kept open. This is especially useful when performing
  automatic, periodic replication.

@@TCcReplicator.LegacyRplTables
Overview: Used only for backwards compatability.
Description: Deprecated. Should always be set to False.

@@TCcReplicator.LocalSYSDBA
Overview: Username and password of the local SYSDBA.
Description: This is only needed for local primary key updates.

@@TCcReplicator.LocalUser
Summary:
Username and password corresponding to the local replication site.
Description:
This is the name of the local replication site. When the replicator logs in to
the remote database, it uses the LocalUser name and password, in order to fetch
all changes applying to the specified user.<p/>
<p/>
The name specified must correspond to a database user on the remote server.
See Also:
<link Replication users>

@@TCcReplicator.Log
Summary:
Replication log.
Description:
Use the Log property to access the TCcLog component used by the replicator.<p/>
<p/>
This component encapsulates the replication log, and provides properties for
accessing its data and manipulating its data. Until the OnLogLoaded event, the
log will be empty.

@@TCcReplicator.OnAbort
Summary:
Fired when the replication is aborted
Description:
OnAbort is fired whenever replication is aborted. This may be because:
         
  1. The application called the AbortReplication method.
  2. A replication error occured, and the application chose to abort (using the
OnReplicationError event and/or the AbortOnError property).
  3. The database connection was lost during processing.
See Also:
Replicate, OnConnectionLost

@@TCcReplicator.OnAutoCommit
Summary:
Fired whenever the replication transactions are auto-committed.
Description:
This event occurs when the AutoCommit property is enabled, and the replication
transactions are automatically committed. See AutoCommit for more information.
See Also:
AutoCommit

@@TCcReplicator.OnCommit
Summary:
Fired just before the transaction is commited at the end of the replication
process.
Description:
This event occurs at the end of the replication, when the CommitOnFinished
property is set to something other than ctNone (that is, ctCommit or
ctRetaining).<p/>
\Note that this is different from the OnAutoCommit event: OnCommit only occurs
when replication is finished, not when performing automatic, timer-based commits.

@@TCcReplicator.OnConflict
Overview: Fired when a conflict is detected.
Description: OnConflict is called
  as soon as a conflict is detected. This does not stop replication of the other records from
  continuing, it simply means that the conflictual record will not be replicated.
SeeAlso:TCcConflictMgr

@@TCcReplicator.OnConnectionLost
Summary:
Fired when a database connection is lost.
Description:
When CopyCat detects that a database connection has been dropped, an
OnConnectionLost event is fired. All remaning database connections are then cut,
and replication is aborted (firing an OnAbort event). This means that if the
remote database becomes temporarily unavailable while replication is in progress,
the application will abort the replication gracefully without throwing any
\exceptions (only events). As soon as the remote database is back online,
replication will be able to resume.

@@TCcReplicator.OnConnectLocal
Summary:
Fired before connecting to the local database.
See Also:
Connect

@@TCcReplicator.OnConnectRemote
Summary:
Fired before connecting to the remote database.
See Also:
Connect

@@TCcReplicator.OnDisconnect
Summary:
Fired before disconnecting from both local and remote database.
Description:
\Note that this event is not fired when the connection is unintentionnaly lost,
but only if the application calls the Disconnect method, or if KeepConnection is
set to False.
See Also:
OnConnectionLost

@@TCcReplicator.OnEmptyLog
Overview: Fired when neither local nor remote replication logs contain any records.

@@TCcReplicator.OnException
Summary:
Fired when an error occurs while no records are being replicated.
Description:
When an error occurs outside the scope of any replication (e.g. when checking for
resolved conflicts, or when loading the log), the OnException event is fired. In
such cases, it's of course impossible to resume replication (as with the
OnReplicationError event), it is simply aborted.
See Also:
Replicate

@@TCcReplicator.OnFinished
Summary:
Fired when the replication process is finished.
Description:
If the log was not empty, an OnFinished event is fired after replicating the
records, even if an error occured and replication was aborted. OnFinished merely
indicates that the replication process is no longer in progress, and can be used
for such things as refreshing the user display and status labels (for example).
See Also:
OnLogLoaded, OnEmptyLog

@@TCcReplicator.OnGenReplError
Summary:
Fired if an error occurs when trying to update the local value of a generator
field.
Description:
This event is fired if an error occurs while updating the value generator field
locally. By implementing the event, you can choose whether to re-try the update,
or to abort. In the latter case, the exception is re-raised, and the
OnReplicationError event is fired.
See Also:
TCcConfig

@@TCcReplicator.OnGenReplicated
Overview: Fired after a generator field was replicated.
Description: This event occurs every time a generator field has been replicated, and
  it provides the name of the field as well as its new value. This occurs once for every
  generator field of a record, but before the fields are updated in the local database.
SeeAlso: TCcConfig

@@TCcReplicator.OnGenReplicating
Overview: Fired just before replicating the generator fields of a record.
Description: This event occurs before the generator fields of the current record get fetched
  from the remote server and applied locally.
SeeAlso: TCcConfig

@@TCcReplicator.OnGetFields
Summary:
Fired after creating the list of fields to be replicated for a table
Description:
This event is called in order to let the developer access, and possibly alter,
the list of fields that are to be replicated for a certain table.<p/>
Every time a new table is encountered in the replication log, the list of
existing fields is loaded from the local database, and (if necessary) from the
remote database also. If Harmonize fields is true, the field list is harmonized,
and then OnGetFields is fired. Thus, the developer can override field list
created by CopyCat.
See Also:
HarmonizeFields

@@TCcReplicator.OnLogLoaded
Summary:
Fired after a non-empty replication log has been loaded
Description:
This event occurs immediately after the log has been loaded, and only if there is
something to replicate. It can therefore be used (for example) for displaying how
many records there are to replicate, and for initializing a progress bar. Before
this point, the log is empty.
See Also:
Replicate, TCcLog

@@TCcReplicator.OnProgress
Summary:
Frequently fired to give the application a chance to refresh the display.
Description:
This event is primarily designed to avoid the application freezing while
replication is in progress. OnProgress is fired after every potentially slow
operation (that is, mainly, remote database operations), and it is therefore a
perfect place to process user-interface messages (using
Application.ProcessMessages for instance), or to otherwise refresh the display.<p/>
<p/>
It is also called every time the current record changes, during replication, so
it can be used for refreshing a progress bar.<p/>

@@TCcReplicator.OnQueryDone
Summary:
Fired for every query that is executing while replicating a record.
Description:
The OnQueryDone event provides the type of query that was executed (SELECT,
UPDATE, INSERT or DELETE), as well as number of rows affected or retrieved.<p/>
<p/>
This allows you to know exactly what database operations were performed, which is
helpful for diagnosing problems, or simply understanding how CopyCat works.

@@TCcReplicator.OnReplicateProc
Overview: Fired every time a stored procedure is replicated.
Description: The OnReplicateProc event occurs every time a stored procedure is replicated,
  providing the name of the procedure. The complete SQL text of the EXECUTE PROCEDURE
  query is available using the PROCEDURE_STATEMENT field of the Log property.
SeeAlso: TCcLog

@@TCcReplicator.OnReplicationError
Summary:
Fired when an error occurs while records are being replicated.
Description:
When an error occurs during replication, the OnReplicationError event is fired,
to determine whether replication may resume (i.e. skip the current record, and
and continue with the subsequent ones), or must be aborted. AbortOnError is used
as a default behaviour in case this event is not implemented.
See Also:
Replicate, OnException

@@TCcReplicator.OnRowReplicated
Overview: Fired after a record has been replicated.

@@TCcReplicator.OnRowReplicating
Overview: Fired upon starting to replicate a new record.

@@TCcReplicator.OnTableBegin
Summary:
Fired upon beginning a new table.
Description:
The lines in the replication log are grouped up by table (due to dependencies
between tables). Therefore, this event is called every time records from a new
table start replicating.
See Also:
OnTableEnd

@@TCcReplicator.OnTableEnd
Summary:
Fired upon finishing a table.
Description:
The lines in the replication log are grouped up by table (due to dependencies
between tables). Therefore, this event is called whenever all the records from a
table have finished replicating.
See Also:
OnTableBegin

@@TCcReplicator.RemoteUser
Summary:
Username and password corresponding to the remote replication site.
Description:
This is the name of the remote replication site. When the replicator logs in to
the local database, it uses the RemoteUser name and password, in order to fetch
all changes applying to the specified user.<p/>
<p/>
The name specified must correspond to a database user on the local server.
See Also:
<link Replication users>

@@TCcReplicator.Replicate
Summary:
Call Replicate to start replicating the two databases.
Description:
If no replication process is already in progress, calling the Replicate method
initiates the replication.<p/>
First a BeforeReplicate event is fired, after which, the database connections are
established, and the replication log is loaded. If the same record is found in
both the remote and the local logs, a conflict is detected, and the OnConflict
event occurs. This does not stop the other records from replicating, it merely
means that the conflictual record will be put "in hold" until the user resolves
the conflict. See TCcConflictMgr for more information about conflict management.<p/>
<p/>
If the log is empty, an OnEmptyLog event is fired. If not, an OnLogLoaded event
is fired, and replication begins. The records to be replicated are grouped up by
table priority, so all the records of the same table replicate together (whether
they from the local or remote databases). Each table is enclosed by the
OnTableBegin and OnTableEnd events, and the OnRowReplicating and OnRowReplicated
events are fired before and after each record. For further detail, the
OnQueryDone event is fired after every query (SELECT, UPDATE, INSERT or DELETE)
that is performed during the replication of one record.<p/>
<p/>
Once replication has started, you may abort it at any time using the
AbortReplication method. If you choose to do so, an OnAbort event will be fired
as soon as the replication process is actually stopped.<p/>
<p/>
If an exception occurs during the replication of a record, the OnReplicationError
event is fired, allowing to either resume replication, or to abort. If an
exception occurs at any other point, an OnException event is fired, and
replication is stopped.<p/>
<p/>
If the log was not empty, an OnFinished event is fired after replicating the
records, even if an error occured and replication was aborted. OnFinished merely
indicates that the replication process is no longer in progress, and can be used
for such things as refreshing the user display and status labels (for example).

@@TCcReplicator.Users
Summary:
User-names and passwords for local, remote and local SYSDBA users.
Description:
The Users property holds the LocalUser, RemoteUser and LocalSYSDBA
sub-properties.
See Also:
<link Replication users>
